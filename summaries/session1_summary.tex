\documentclass{article}
\usepackage{graphicx} % Required for inserting images
 \usepackage{amsmath}
\title{ECI Internal Training \\ 
First Session}
% \author{Thomas Lorans}
\date{November 2025}

\begin{document}

\maketitle

\section{Introduction}

\begin{itemize}
    \item Asset pricing = present value of future cash flows.
    \item Two building blocks:
    \begin{itemize}
        \item Time: discounting, intertemporal substitution.
        \item Risk: states of the world, contingent claims.
    \end{itemize}
    \item Use expected utility: $E[u(c)]$.
\end{itemize}


\section{Microfoundations of Asset Pricing Theory}


\subsection{Few Recap on Optimization}

%=========================================================
\paragraph{Unconstrained optimization: $\max_x F(x)$, FOC $F'(x^*)=0$, SOC $F''(x^*)<0$.}
\begin{itemize}
    \item  Function:
    \[
        F(x) = -\frac{1}{2}(x-5)^2.
    \]
    \item First derivative:
    \[
        F'(x) = -(x-5).
    \]
    \item Solve FOC:
    \[
        F'(x^*) = 0 \quad\Rightarrow\quad x^* = 5.
    \]
    \item Second derivative:
    \[
        F''(x) = -1 < 0,
    \]
    so the SOC confirms a strict maximum at \(x=5\).
\end{itemize}

%=========================================================
\paragraph{Concavity: $F''(x)<0$ for all $x$ $\Rightarrow$ FOC is necessary and sufficient.}
\begin{itemize}
    \item Function:
    \[
        F(x) = -\frac{1}{2}(x-5)^2.
    \]
    \item Second derivative:
    \[
        F''(x) = -1 < 0 \quad \forall x.
    \]
    \item Since $F$ is globally concave, *any* solution to $F'(x)=0$ is automatically the global maximizer.
\end{itemize}

%=========================================================
\paragraph{Constrained problem: $\max_x F(x)$ s.t. $c \ge G(x)$.}
\begin{itemize}
    \item non-binding:
    \[
        \max_x -\frac{1}{2}(x-5)^2 \quad \text{s.t.}\quad 7 \ge x.
    \]
    \item Unconstrained optimum: $x^*=5$.
    \item Check constraint: $7 > 5$ → constraint non-binding → optimal solution unchanged: $x^*=5$.
\end{itemize}

%=========================================================
\paragraph{Lagrangian: $L(x,\lambda)=F(x)+\lambda(c-G(x))$.}
\begin{itemize}
    \item Lagrangian:
    \[
        L(x,\lambda)= -\frac{1}{2}(x-5)^2 + \lambda(7-x).
    \]
    \item First-order condition:
    \[
        \frac{\partial L}{\partial x}= -(x-5) - \lambda = 0.
    \]
    \item Complementary slackness:
    \[
        \lambda(7-x)=0.
    \]
    \item Non-binding case: $x=5$, $\lambda=0$ satisfies both.
\end{itemize}

%=========================================================
\paragraph{KKT conditions.}
\begin{itemize}
    \item Maximization:
    \[
        \max_x -\frac{1}{2}(x-5)^2 \quad \text{s.t.}\quad 4 \ge x.
    \]
    \item Lagrangian:
    \[
        L(x,\lambda)= -\frac{1}{2}(x-5)^2 + \lambda(4-x).
    \]
    \item First-order condition:
    \[
        -(x-5) - \lambda = 0.
    \]
    \item Complementary slackness:
    \[
        \lambda(4-x)=0.
    \]
    \item Since unconstrained optimum $x=5$ is infeasible, constraint must bind: $x^*=4$.
    \item Solve for $\lambda$:
    \[
        -(4-5) - \lambda = 0 
        \quad\Rightarrow\quad 
        1 - \lambda = 0 
        \quad\Rightarrow\quad 
        \lambda = 1.
    \]
    \item Final solution:
    \[
        x^*=4,\qquad \lambda^*=1.
    \]
\end{itemize}


\subsection{The Time Dimension}

%===================================================
\paragraph{Utility: $u(c_0)+\beta u(c_1)$.}
\begin{itemize}
    \item As in the slides, reinterpret the two goods as:
    \[
        c_0 = \text{consumption today},\qquad
        c_1 = \text{consumption next year}.
    \]
    \item Fisher’s two–period intertemporal utility:
    \[
        U = u(c_0) + \beta u(c_1),
    \]
    where $\beta$ is the discount factor (slide 68).
    \item Concavity of $u$ implies a preference for smooth consumption (slide 69).
\end{itemize}

%===================================================
\paragraph{Budget constraints.}
\begin{itemize}
    \item Today’s tradeoff (slide 71):
    \[
        Y_0 \ge c_0 + s,
    \]
    where $s$ is saving (or borrowing if $s<0$).
    \item Next year’s constraint (slide 71):
    \[
        Y_1 + (1+r)s \ge c_1.
    \]
    \item Saving increases next year’s consumption; borrowing reduces it.
\end{itemize}

%===================================================
\paragraph{Lifetime budget constraint.}
\begin{itemize}
    \item Divide next year’s constraint by $(1+r)$ (slide 72):
    \[
        \frac{Y_1}{1+r} + s \ge \frac{c_1}{1+r}.
    \]
    \item Combine with the first-period constraint $Y_0 \ge c_0 + s$:
    \[
        Y_0 + \frac{Y_1}{1+r}
        \ge
        c_0 + \frac{c_1}{1+r}.
    \]
    \item This is the *lifetime* budget constraint derived on slide 73:
    the present value of consumption cannot exceed the present value of income.
\end{itemize}

%===================================================
\paragraph{Intertemporal marginal rate of substitution.}
\begin{itemize}
    \item Lagrangian for the time-allocation problem (slide 77):
    \[
        L = u(c_0) + \beta u(c_1)
        + \lambda\!\left(
            Y_0 + \frac{Y_1}{1+r}
            - c_0 - \frac{c_1}{1+r}
        \right).
    \]
    \item First-order conditions (slide 78):
    \[
        u'(c_0) - \lambda = 0,\qquad
        \beta u'(c_1) - \lambda\!\left(\frac{1}{1+r}\right)=0.
    \]
    \item Eliminate $\lambda$ using $\lambda = u'(c_0)$:
    \[
        \beta u'(c_1) = \frac{u'(c_0)}{1+r}.
    \]
    \item Rearrange to obtain the intertemporal MRS (slide 78):
    \[
        \frac{u'(c_0)}{\beta u'(c_1)} = 1+r.
    \]
    \item Interpretation:  
    MRS (marginal willingness to give up $c_1$ for $c_0$)  
    equals the relative price of consumption across periods.
\end{itemize}
\subsection{The Risk Dimension}

%===================================================
\paragraph{Setup: Utility under uncertainty.}
\begin{itemize}
    \item As in slides (86–91), income next year is state-dependent:
    \[
        Y_1^G \quad\text{(good state)},\qquad
        Y_1^B \quad\text{(bad state)},\qquad
        Y_1^G > Y_1^B.
    \]
    \item Probabilities:
    \[
        \pi = P(G),\qquad 1-\pi = P(B).
    \]
    \item Consumption choices:
    \[
        c_0,\quad c_1^G,\quad c_1^B.
    \]
    \item Expected utility (slide 91):
    \[
        U = u(c_0) + \beta\big[\pi u(c_1^G) + (1-\pi)u(c_1^B)\big].
    \]
    \item Concavity of \(u\) captures risk aversion and desire for smoothness across states (slide 94).
\end{itemize}

%===================================================
\paragraph{Budget constraints and contingent claims.}
\begin{itemize}
    \item The consumer can buy contingent claims (slides 95–101):
    \[
        q_G: \text{ price of claim paying 1 in good state},\qquad
        q_B: \text{ price of claim paying 1 in bad state}.
    \]
    \item Purchases:
    \[
        s_G,\, s_B \quad \text{(negative = shorting)}.
    \]
    \item Budget today (slide 101):
    \[
        Y_0 \ge c_0 + q_G s_G + q_B s_B.
    \]
    \item State-by-state constraints (slide 102):
    \[
        Y_1^G + s_G \ge c_1^G,\qquad
        Y_1^B + s_B \ge c_1^B.
    \]
\end{itemize}

%===================================================
\paragraph{Deriving the lifetime budget constraint.}
\begin{itemize}
    \item Multiply the good-state constraint by \(q_G\) and the bad-state constraint by \(q_B\) (slide 103):
    \[
        q_G Y_1^G + q_G s_G \ge q_G c_1^G,
    \]
    \[
        q_B Y_1^B + q_B s_B \ge q_B c_1^B.
    \]
    \item Add these and combine with today’s constraint:
    \[
        Y_0 + q_G Y_1^G + q_B Y_1^B
        \ge
        c_0 + q_G c_1^G + q_B c_1^B.
    \]
    \item This is the lifetime Arrow–Debreu budget constraint (slide 103).
\end{itemize}

%===================================================
\paragraph{FOCs and marginal rates of substitution.}
\begin{itemize}
    \item Lagrangian (slide 105):
    \[
        L = u(c_0) + \beta\pi u(c_1^G) + \beta(1-\pi)u(c_1^B)
        + \lambda\!\left(
            Y_0 + q_G Y_1^G + q_B Y_1^B
            - c_0 - q_G c_1^G - q_B c_1^B
        \right).
    \]
    \item First-order conditions (slide 105):
    \[
        u'(c_0) - \lambda = 0,
    \]
    \[
        \beta\pi u'(c_1^G) - \lambda q_G = 0,
    \]
    \[
        \beta(1-\pi)u'(c_1^B) - \lambda q_B = 0.
    \]
    \item Eliminate \(\lambda\) to get the pricing relations (slide 106):
    \[
        \frac{u'(c_0)}{\beta \pi u'(c_1^G)} = \frac{1}{q_G},
        \qquad
        \frac{u'(c_0)}{\beta (1-\pi) u'(c_1^B)} = \frac{1}{q_B}.
    \]
    \item State-relative price condition (slide 106):
    \[
        \frac{\pi u'(c_1^G)}{(1-\pi)u'(c_1^B)}
        = \frac{q_G}{q_B}.
    \]
\end{itemize}

%===================================================
\paragraph{Interpreting contingent claims as building blocks.}
\begin{itemize}
    \item A stock with dividends \((d_G, d_B)\) is equivalent to a bundle of contingent claims (slides 108–112):
    \[
        q_{\text{stock}} = q_G d_G + q_B d_B.
    \]
    \item A risk-free bond paying \(1\) in every state (slide 114):
    \[
        q_{\text{bond}} = q_G + q_B = \frac{1}{1+r}.
    \]
    \item So the risk-free rate is (slide 114):
    \[
        1+r = \frac{1}{q_G + q_B}.
    \]
\end{itemize}

%===================================================
\paragraph{Recovering contingent claim prices from traded assets.}
\begin{itemize}
    \item Using a stock (\(d_G, d_B\)) and a bond, we can solve for \(q_G\), \(q_B\) via replication (slides 117–122).
    \item To replicate a good-state claim: solve
    \[
        s d_G + b = 1,\qquad s d_B + b = 0.
    \]
    \item Solution (slide 118):
    \[
        s = \frac{1}{d_G - d_B},\qquad
        b = \frac{-d_B}{d_G - d_B}.
    \]
    \item Hence (slide 119):
    \[
        q_G = q_{\text{stock}}\,s + q_{\text{bond}}\,b.
    \]
    \item Similarly for the bad-state claim (slide 122):
    \[
        q_B = \frac{d_G q_{\text{bond}} - q_{\text{stock}}}{d_G - d_B}.
    \]
\end{itemize}

\subsection{General Equilibrium}

%===================================================
\paragraph{Pareto optimality condition.}
\begin{itemize}
    \item In the Edgeworth box (slides 126–132), two consumers trade goods $a$ and $b$.
    \item A feasible allocation is Pareto optimal when no reallocation can make one consumer better off without making the other worse off.
    \item Geometrically: indifference curves are tangent (slide 131).
    \item Mathematically (slide 132):
    \[
        MRS^1_{a,b} = MRS^2_{a,b}.
    \]
    \item This states that the two consumers’ marginal rates of substitution must coincide.
\end{itemize}

%===================================================
\paragraph{Social planner conditions.}
\begin{itemize}
    \item Utilities from slides (133–135):
    \[
        U_1 = u(c_a^1) + \alpha u(c_b^1), \qquad
        U_2 = v(c_a^2) + \beta v(c_b^2).
    \]
    \item Resource constraints:
    \[
        Y_a \ge c_a^1 + c_a^2,\qquad
        Y_b \ge c_b^1 + c_b^2.
    \]
    \item Social planner maximizes weighted sum (slide 134):
    \[
        \theta U_1 + (1-\theta)U_2.
    \]
    \item Lagrangian (slide 135):
    \[
        L=\theta[u(c_a^1)+\alpha u(c_b^1)]
        +(1-\theta)[v(c_a^2)+\beta v(c_b^2)]
        +\lambda_a(Y_a - c_a^1 - c_a^2)
        +\lambda_b(Y_b - c_b^1 - c_b^2).
    \]
    \item First-order conditions (slide 135):
    \[
        \theta u'(c_a^1) = \lambda_a,\qquad
        \theta \alpha u'(c_b^1) = \lambda_b,
    \]
    \[
        (1-\theta) v'(c_a^2) = \lambda_a,\qquad
        (1-\theta)\beta v'(c_b^2) = \lambda_b.
    \]
    \item Divide each consumer’s $a$–equation by the $b$–equation (slide 136):
    \[
        \frac{u'(c_a^1)}{\alpha u'(c_b^1)}
        =
        \frac{v'(c_a^2)}{\beta v'(c_b^2)}.
    \]
    \item This is exactly the Pareto condition:
    \[
        MRS^1_{a,b} = MRS^2_{a,b}.
    \]
\end{itemize}

%===================================================
\paragraph{Competitive equilibrium condition.}
\begin{itemize}
    \item Consumers maximize utility subject to market prices (slides 138–142).
    \item Consumer 1’s Lagrangian (slide 139):
    \[
        L_1 = u(c_a^1) + \alpha u(c_b^1) 
        +\lambda_1\big(p_a Y_a^1 + p_b Y_b^1
        - p_a c_a^1 - p_b c_b^1\big).
    \]
    \item FOCs:
    \[
        u'(c_a^1)=\lambda_1 p_a,\qquad
        \alpha u'(c_b^1)=\lambda_1 p_b.
    \]
    \item Therefore (slide 139):
    \[
        \frac{u'(c_a^1)}{\alpha u'(c_b^1)}=\frac{p_a}{p_b}.
    \]
    \item Similarly for consumer 2 (slide 141):
    \[
        \frac{v'(c_a^2)}{\beta v'(c_b^2)}=\frac{p_a}{p_b}.
    \]
    \item Hence in any competitive equilibrium (slide 142):
    \[
        MRS^1_{a,b} = \frac{p_a}{p_b} = MRS^2_{a,b}.
    \]
\end{itemize}

%===================================================
\paragraph{Welfare theorems.}
\begin{itemize}
    \item Slide 145 formalizes the main results:
    \begin{itemize}
        \item \textbf{First Welfare Theorem:}  
        Any competitive equilibrium allocation satisfies the PO condition:  
        \[
        CE \;\Rightarrow\; PO.
        \]
        \item \textbf{Second Welfare Theorem:}  
        Any Pareto optimal allocation can be supported as a competitive equilibrium with appropriate prices:  
        \[
        PO \;\Rightarrow\; CE \;\text{for some prices}.
        \]
    \end{itemize}
\end{itemize}

\section{Overview of Asset Pricing Theory}

%============================================================
\subsection{Pricing Safe Cash Flows}

\paragraph{Discount bond pricing.}
\begin{itemize}
    \item A $T$-year discount bond pays \$1 at maturity (slide 3).
    \item If its price today is $P_T$, the gross return is:
    \[
        1+r_T = \left(\frac{1}{P_T}\right)^{1/T}.
    \]
    \item Therefore (slide 3):
    \[
        P_T = \frac{1}{(1+r_T)^T}.
    \]
    \item Interpretation: price = present discounted value of a certain future payment.
\end{itemize}

\paragraph{Coupon bond pricing.}
\begin{itemize}
    \item Coupon bond pays $C$ each year for $T$ years and face value $F$ at maturity (slide 5).
    \item Decomposed as portfolio of discount bonds (slide 6).
    \item No-arbitrage implies (slide 8):
    \[
        P_T^C
        = \frac{C}{1+r_1}
        + \frac{C}{(1+r_2)^2}
        + \cdots
        + \frac{C}{(1+r_T)^T}
        + \frac{F}{(1+r_T)^T}.
    \]
    \item The yield to maturity $r$ solves (slide 9):
    \[
        P_T^C 
        = \sum_{t=1}^T \frac{C}{(1+r)^t}+\frac{F}{(1+r)^T}.
    \]
\end{itemize}

\paragraph{General stream of safe cash flows.}
\begin{itemize}
    \item For riskless cash flows $C_1,\dots,C_T$ (slide 11):
    \[
        P = \sum_{t=1}^T C_t P_t.
    \]
    \item Since $P_t = \frac{1}{(1+r_t)^t}$ (slide 12):
    \[
        P = \sum_{t=1}^T \frac{C_t}{(1+r_t)^t}.
    \]
\end{itemize}

%============================================================
\subsection{Pricing Risky Cash Flows}

\paragraph{Expected value approach.}
\begin{itemize}
    \item Random payoff $\tilde{C}_t$ with possible outcomes $\{C_{t,i}\}$ and probabilities $\{\pi_i\}$ (slide 15):
    \[
        E(\tilde{C}_t)=\sum_i \pi_i C_{t,i}.
    \]
    \item Risk-adjusted discounting (slide 16):
    \[
        P_t^A = \frac{E(\tilde{C}_t)}{(1+r_t+\psi_t)^t},
    \]
    or equivalently
    \[
        P_t^A = \frac{E(\tilde{C}_t)-\Psi_t}{(1+r_t)^t}.
    \]
    \item CAPM, CCAPM, APT provide formulas for $\psi_t$ or $\Psi_t$ (slide 17).
\end{itemize}

\paragraph{Arrow–Debreu (contingent claim) pricing.}
\begin{itemize}
    \item Decompose payoff into state components (slide 18):
    \[
        \tilde{C}_t = (C_{t,1},\dots,C_{t,n}).
    \]
    \item With contingent claim prices $q_{t,i}$ (slide 19):
    \[
        P_t^A = \sum_{i=1}^n q_{t,i} C_{t,i}.
    \]
    \item Interpretation: the risky asset = portfolio of state-contingent discount bonds.
\end{itemize}

\paragraph{Martingale / risk-neutral pricing.}
\begin{itemize}
    \item “Distort” probabilities: replace $\pi_i$ by $\hat{\pi}_i$ (slide 20).
    \item Compute distorted expectation:
    \[
        \hat{E}(C_t) = \sum_i \hat{\pi}_i C_{t,i}.
    \]
    \item Price using risk-free discounting:
    \[
        P_t^A = \frac{\hat{E}(C_t)}{(1+r_t)^t}.
    \]
    \item This is the basis of martingale pricing and risk-neutral valuation.
\end{itemize}

%============================================================
\subsection{Two Perspectives on Asset Pricing}

\paragraph{No-arbitrage perspective.}
\begin{itemize}
    \item Takes some prices as given and derives others (slide 21).
    \item Examples in the slides:
    \begin{itemize}
        \item Coupon bond = portfolio of discount bonds.
        \item Risky payoff = portfolio of contingent claims.
    \end{itemize}
    \item Requires only that arbitrage opportunities do not exist.
\end{itemize}

\paragraph{Equilibrium perspective.}
\begin{itemize}
    \item Uses microeconomic foundations: consumers maximize expected utility; markets clear (slide 21).
    \item Determines all prices simultaneously.
    \item Links asset prices to fundamentals such as marginal rates of substitution.
\end{itemize}

\paragraph{Unified view.}
\begin{itemize}
    \item Slide 24 summary:
    \[
        \begin{array}{lll}
            \text{Model} & \text{Equilibrium} & \text{No-Arbitrage} \\
            \hline
            \text{Risk Premia} & \text{CAPM, CCAPM} & \text{APT} \\
            \text{Contingent Claims} & \text{A--D} & \text{A--D} \\
            \text{Distorted Probabilities} & \text{Martingale} &   
        \end{array}
    \]
    \item All models aim to compute 
    \[
        P_0 = \frac{\text{risk-adjusted payoff}}{\text{risk-free discount factor}}.
    \]
\end{itemize}


\end{document}
